\documentclass[11pt,fleqn]{article}
\usepackage[margin=1in]{geometry}
\usepackage{tikz}
\usepackage{mathtools}
\usepackage{longtable}
\usepackage{enumitem}
\usepackage[colorlinks = true,
		linkcolor=black,
		citecolor=black,
	        urlcolor  = black]{hyperref}
%\usepackage[dvips]{graphics}
%\usepackage[table]{xcolor}
%\usepackage{amssymb}
\usepackage{float}
%\usepackage{subfig}
\usepackage{booktabs}
\usepackage{subcaption}
\usepackage{booktabs}

\usepackage[normalem]{ulem}

\usepackage{multicol}
\usepackage{txfonts}
\usepackage{amsfonts}
\usepackage{natbib}

\usepackage{multirow}

\usepackage{gb4e}
\usepackage[all]{xy}
\usepackage{rotating}
\usepackage{tipa}
\usepackage{multirow}
\usepackage{authblk}
\usepackage{url}
\usepackage{pdflscape}
\usepackage{rotating}
\usepackage{adjustbox}
\usepackage{array}

\usepackage{dcolumn} % for printing model output tables directly from R


%\usepackage{color}
%\DeclareOuterCiteDelims{cite}{\textcolor{green}{\bibopenbracket}}{\textcolor{red}{\bibclosebracket}}

\definecolor{Pink}{RGB}{255,50,170}
\newcommand{\jd}[1]{\textcolor{Pink}{[jd: #1]}}  

\newcommand{\jt}[1]{\textbf{\color{purple}JT: #1}}

\newcommand{\tableref}[1]{Tab.~\ref{#1}}
\newcommand{\figref}[1]{Fig.~\ref{#1}}

\def\bad{{\leavevmode\llap{*}}}
\def\marginal{{\leavevmode\llap{?}}}
\def\verymarginal{{\leavevmode\llap{??}}}
\def\swmarginal{{\leavevmode\llap{4}}}
\def\infelic{{\leavevmode\llap{\#}}}

\definecolor{airforceblue}{rgb}{0.36, 0.54, 0.66}
%\definecolor{gray}{rgb}{0.36, 0.54, 0.66}

\newcommand{\dashrule}[1][black]{%
  \color{#1}\rule[\dimexpr.5ex-.2pt]{4pt}{.4pt}\xleaders\hbox{\rule{4pt}{0pt}\rule[\dimexpr.5ex-.2pt]{4pt}{.4pt}}\hfill\kern0pt%
}

\setlength{\parindent}{.3in}
\setlength{\parskip}{0ex}

\newcommand{\yi}{\'{\symbol{16}}}
\newcommand{\nasi}{\~{\symbol{16}}}
\newcommand{\hina}{h\nasi na}
\newcommand{\ina}{\nasi na}

\newcommand{\foc}{$_{\mbox{\small F}}$}

\hyphenation{par-ti-ci-pa-tion}

%\setlength{\bibhang}{0.5in}
%\setlength{\bibsep}{0mm}
%\bibpunct[:]{(}{)}{,}{a}{}{,}

\newcommand{\6}{\mbox{$[\hspace*{-.6mm}[$}} 
\newcommand{\9}{\mbox{$]\hspace*{-.6mm}]$}}
\newcommand{\sem}[2]{\6#1\9$^{#2}$}
\renewcommand{\ni}{\~{\i}}

\newcommand{\citepos}[1]{\citeauthor{#1}'s \citeyear{#1}}
\newcommand{\citeposs}[1]{\citeauthor{#1}'s}
\newcommand{\citetpos}[1]{\citeauthor{#1}'s \citeyear{#1}}

\newcolumntype{R}[2]{%
    >{\adjustbox{angle=#1,lap=\width-(#2)}\bgroup}%
    l%
    <{\egroup}%
}
\newcommand*\rot{\multicolumn{1}{R{90}{0em}}}% no optional argument here, please!

\title{Hypotheses}

\author{YK and JT}

%\author[$\bullet$]{Judith Degen}
%\author[$\circ$]{Judith Tonhauser}
%
%\affil[$\bullet$]{Stanford University}
%\affil[$\circ$]{University of Stuttgart}
%
%\renewcommand\Authands{ and }

\begin{document}

\maketitle

\citealt{kiparsky-kiparsky70}

\begin{exe}
\ex here is a first example
\ex here is another one
\begin{xlist}
\ex another
\ex one
\end{xlist}
\ex and a third one
\end{exe}

\section{H1: Emotive vs.\ doxastic vs.\ communicative vs.\ inferential}

\subsection{Classification of predicates}

All the predicates occur in the past tense in the MV dataset, so we also used the past tense to code up the lexical meaning.

\begin{itemize}

\item Communicative: A predicate P is communicative if and only if ``X Ped that m" requires X to have externalized that m is the case/on the table. The externalization may be have been verbal or nonverbal. 

\begin{itemize}

\item Pure: ``say'' (you can do this on your own)

\item Discourse participation: ``deny, respond'' (you can't do this on your own, requires another interlocutor)

\item State changing: ``demonstrate, prove, fake, conceal'' (you can't do this on your own, AH's communicative act is combined with the intention to change somebody's belief state)

\end{itemize}

\item Private: A predicate P is private if and only if ``X Ped that m" conveys that m stands in some relation to X's mental representation of the world (which doesn't require X to believe that m is true).

\begin{itemize}

\item Emotive: ``be amused, feel'', X has a feeling or emotion towards m

\item Cognitive: ``think, know, discover'', conveys something about X's relation to m

\begin{itemize}

\item Stative: ``think, know, deluded''

\item Telic: ``discover, realize''

\item Activity: ``contemplate, reminisce'' 

\end{itemize}

\item Evidential: A predicate P is evidential if and only if ``X Ped that m'' conveys the source of information by which X received the information about p.

``was bet'': private, evidential (reportative)

``was challenged'': REL(X,m)

``was chastized, was congratulated, was consulted (informed)'': private, evidential, reportative

``was deplored that'': only occurs with periphrastic ``it''

``was forgiven that'': REL(X,m)

``was jaded that'': emotive

``listened that'': not categorized

\begin{itemize}

\item Pure: ``X saw that p, X heard that p, X reasoned/realized that p'' (inferential may be conjectural/indirect evidence)

\item Passivized: ``was told that, was (mis)informed that', was contacted that'

\end{itemize}

\end{itemize}

\end{itemize}

\subsection{Projection of different predicate types}

Visualisation of mean projection ratings by predicate and predicate type (communicative, emotive, cognitive, evidential) shows higher mean projection ratings for emotives than for other predicate types. Visualisation of mean projection ratings by predicate type and voice shows higher mean projection ratings for active voice cognitives and evidentials compared to passive predicates of these types. It further shows higher mean projection ratings for passive emotives than for active ones. 

\subsubsection{Projection of verbal and adjectival emotives}

The emotive predicates labelled ``passive'' in the MV data set are not passivised verbal predicates, but adjectival predicates. Visualisation of the six emotive predicates which occur both as verbal and adjectival predicates in the MV data set suggests that the difference in mean projection ratings between verbal and adjectival emotives may not be significant. Visualisation of mean projection ratings of 402 verbal predicates shows that whilst many of the remaining 27 emotives are amongst those predicates with the highest mean projection ratings, this overall pattern is less pronounced than the investigation based on all 544 predicates suggested.

\section{Other hypotheses}


- state vs event

- gradability



\bibliographystyle{../cslipubs-natbib}
\bibliography{../bibliography}


\end{document}
